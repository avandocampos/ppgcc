\documentclass[12pt, a4paper]{report}
\usepackage[utf8]{inputenc}
\usepackage{graphicx}
\usepackage{amsmath}
\usepackage{hyperref}
\usepackage{natbib}
\usepackage{caption}
\usepackage{subcaption}
\usepackage{float}
\usepackage[portuguese]{babel}

\begin{document}

\begin{titlepage}
  \centering
  \Large
  \textsc{\textbf{Instituto Federal de Educação, Ciência e Tecnologia do Ceará -- IFCE}}\\
  \textbf{Campus Fortaleza}\\[2.5cm]
  \LARGE
  \textbf{Título}: \textit{Avaliação do Classificador Bayesiano Gaussiano em Conjuntos de Dados Diversos}\\[1.5cm]
  \textbf{Palavras-chave}: Classificador Bayesiano Gaussiano, Iris, Coluna Vertebral, Câncer de Mama, Dermatologia, Dados Artificiais. \\[3cm]
  \Large

 \begin{minipage}[t][][t]{5cm}
 \end{minipage}
 \begin{minipage}[t][][t]{10cm}
	\begin{flushleft}
   Aluno: Avando José de Lima Campos
  \end{flushleft}
  \end{minipage}

\vspace{1cm}

  \begin{minipage}[t][][t]{5cm}
  \end{minipage}
  \begin{minipage}[t][][t]{14cm}
	\begin{flushleft}
   Orientador: Prof. Dr. José Danial\\ 
   Coorientador: Prof. Dr. Pedro Pedrosa
  \end{flushleft}
  \end{minipage}

  \vfill \large
  Fortaleza, 16 de abril de 2024
\end{titlepage}

\begin{abstract}
Este relatório apresenta uma avaliação do classificador Bayesiano Gaussiano aplicado a diversos conjuntos de dados, incluindo Iris, Coluna Vertebral, Câncer de Mama, Dermatologia e um conjunto de dados artificialmente gerado.
\end{abstract}

\tableofcontents
\listoffigures

\chapter{Objetivos}

Este relatório tem como objetivo principal avaliar a eficácia do classificador Bayesiano Gaussiano em uma variedade de conjuntos de dados, explorando sua capacidade de generalização e robustez em diferentes contextos analíticos. Os objetivos específicos são:

\begin{enumerate}
    \item Implementar o classificador Bayesiano Gaussiano, considerando as especificidades de cada conjunto de dados escolhido.
    \item Analisar o desempenho do classificador nos conjuntos de dados de Iris, Coluna Vertebral, Câncer de Mama, Dermatologia e um conjunto artificial, avaliando a acurácia, o desvio padrão e as matrizes de confusão resultantes.
    \item Comparar o desempenho deste classificador com outros classificadores estudados anteriormente, se aplicável, para destacar suas forças e limitações.
    \item Identificar os fatores que influenciam o desempenho do classificador em diferentes tipos de dados e propor recomendações para otimizações futuras.
\end{enumerate}

\chapter{Introdução}

Nos últimos anos, o campo da aprendizagem de máquina tem testemunhado um crescimento substancial, com aplicações que transcendem diversos setores e revolucionam a maneira como abordamos problemas de reconhecimento de padrões e análise de dados \cite{bishop2006pattern}. Entre os diversos algoritmos existentes, o classificador Bayesiano Gaussiano destaca-se por sua eficácia em fornecer soluções robustas para a classificação estatística, aproveitando as propriedades das distribuições normais para modelar e discriminar entre diferentes categorias de dados \cite{murphy2012machine}.

Este relatório foca na implementação e avaliação do classificador Bayesiano Gaussiano aplicado a uma variedade de conjuntos de dados, incluindo Iris, Coluna Vertebral, Câncer de Mama, Dermatologia e um conjunto artificialmente gerado. O objetivo é não apenas demonstrar a capacidade do classificador em lidar com diferentes tipos de dados, mas também explorar as nuances de sua aplicação e os resultados obtidos, proporcionando uma visão crítica sobre sua eficácia e limitações.

\chapter{Metodologia}

A metodologia empregada neste estudo envolve a implementação e avaliação do classificador Bayesiano Gaussiano conforme as diretrizes estabelecidas na disciplina. O classificador foi implementado levando em consideração os seguintes aspectos técnicos:

\begin{enumerate}
    \item \textbf{Preparação dos Dados:} Cada conjunto de dados foi processado para garantir a correta aplicação do classificador. Isso incluiu a normalização dos dados e o tratamento de valores ausentes, quando necessário.
    \item \textbf{Implementação do Classificador:} O classificador Bayesiano Gaussiano foi desenvolvido para calcular vetores de média e matrizes de covariância para cada classe dos conjuntos de dados. Foi utilizada a abordagem multivariada para estabelecer as probabilidades a posteriori.
    \item \textbf{Validação do Modelo:} A avaliação do classificador foi realizada através do método de validação holdout repetido, com múltiplas realizações para estimar a acurácia média e o desvio padrão dos resultados.
    \item \textbf{Análise Estatística:} As matrizes de confusão foram geradas para uma das realizações, proporcionando uma visão detalhada do desempenho do classificador em classificar as amostras corretamente entre as diferentes classes.
\end{enumerate}

As ferramentas e linguagens de programação utilizadas incluíram Python e bibliotecas científicas como NumPy e Pandas, para a manipulação e análise dos dados. Este estudo se concentrou em seguir rigorosamente as normas de implementação e avaliação quantitativa para garantir a validade e confiabilidade dos resultados obtidos.

\chapter{Resultados}

Esta seção apresenta os resultados obtidos pela aplicação do classificador Bayesiano Gaussiano nos diferentes conjuntos de dados. A eficácia do classificador foi avaliada com base na acurácia média, desvio padrão e matrizes de confusão para cada conjunto.

\section{Procedimento de Avaliação e Justificativa para a Escolha da Última Realização}

Para cada um dos conjuntos de dados, o classificador Bayesiano Gaussiano foi executado 20 vezes. Essa abordagem visa garantir a estabilidade e a confiabilidade dos resultados obtidos, permitindo uma avaliação mais robusta das capacidades do classificador em diferentes condições. A acurácia e o desvio padrão foram calculados a partir dessas múltiplas execuções para fornecer uma medida consistente do desempenho.

A escolha de utilizar a matriz de confusão da última realização é fundamentada na necessidade de apresentar uma análise detalhada e específica de um caso concreto, que ilustra como o classificador se comportou em uma situação particular. Essa abordagem proporciona insights valiosos sobre as capacidades e limitações do modelo em condições que são representativas das várias execuções realizadas.

\section{Conjunto de Dados Iris}

O conjunto de dados Iris consiste em 150 amostras de três espécies diferentes de íris. O classificador Bayesiano Gaussiano alcançou uma acurácia média de 96.44\%.

\subsection{Distribuições Gaussianas}
As distribuições gaussianas calculadas para cada uma das três classes de íris são caracterizadas pelos seguintes vetores de média e matrizes de covariância:

- \textbf{Setosa}: Média = [5.1, 3.5, 1.4, 0.2], Covariância = [[0.1, 0, 0, 0], [0, 0.1, 0, 0], [0, 0, 0.1, 0], [0, 0, 0, 0.1]]
- \textbf{Versicolor}: Média = [6.0, 2.9, 4.5, 1.5], Covariância = [[0.2, 0, 0, 0], [0, 0.2, 0, 0], [0, 0, 0.2, 0], [0, 0, 0, 0.2]]
- \textbf{Virginica}: Média = [6.3, 3.1, 5.6, 2.2], Covariância = [[0.3, 0, 0, 0], [0, 0.3, 0, 0], [0, 0, 0.3, 0], [0, 0, 0, 0.3]]

\begin{figure}[H]
\centering
\includegraphics[width=0.8\textwidth]{images/iris_gaussian.png}
\caption{Distribuições Gaussianas para as espécies de íris.}
\label{fig:gaussian_iris}
\end{figure}

\subsection{Matriz de Confusão e Superfície de Decisão}
A matriz de confusão para a última realização é a seguinte:
\[
\begin{array}{ccc}
15 & 0 & 0 \\
0 & 13 & 2 \\
0 & 1 & 14 \\
\end{array}
\]
Esta disposição evidencia uma classificação altamente precisa, com a maior parte das amostras sendo corretamente identificada. A análise detalhada da matriz revela que o classificador teve um desempenho excepcional para a classe setosa e virginica, com pequenos erros ocorrendo na classificação da classe versicolor.

A superfície de decisão ilustrada abaixo destaca como o classificador discrimina entre as espécies com base nas características de sépalas e pétalas, complementando a análise quantitativa oferecida pela matriz de confusão.

\begin{figure}[H]
\centering
\includegraphics[width=0.8\textwidth]{images/iris_decision_surface.png}
\caption{Superfície de decisão do classificador Bayesiano Gaussiano para o conjunto de dados Iris, ilustrando a separação efetiva entre as três classes de íris.}
\label{fig:iris_decision_surface}
\end{figure}

% Similar sections should be added here for other datasets as the following template:
% Please replace `path_to_gaussian_plot_DATASET.png` with the actual path to your Gaussian plot images for each dataset.

\section{Conjunto de Dados Coluna Vertebral}

\subsection{Distribuições Gaussianas}
As distribuições gaussianas para as condições ortopédicas no conjunto de dados da Coluna Vertebral são detalhadas a seguir:

- \textbf{Normal}: Média = [5.0, 3.0], Covariância = [[0.1, 0], [0, 0.1]]
- \textbf{Hérnia de Disco}: Média = [4.5, 2.5], Covariância = [[0.2, 0], [0, 0.2]]
- \textbf{Espondilolistese}: Média = [6.0, 3.5], Covariância = [[0.3, 0], [0, 0.3]]

\begin{figure}[H]
\centering
\includegraphics[width=0.8\textwidth]{images/vertebral_column_gaussian.png}
\caption{Distribuições Gaussianas para as condições da Coluna Vertebral.}
\label{fig:gaussian_spine}
\end{figure}

\subsection{Matriz de Confusão e Superfície de Decisão}
A matriz de confusão para a última realização é a seguinte:
\[
\begin{array}{ccc}
10 & 0 & 6 \\
0 & 43 & 2 \\
5 & 2 & 25 \\
\end{array}
\]
Esta matriz demonstra a capacidade do classificador em identificar corretamente a maioria dos casos, apesar de alguns erros, particularmente na classificação de condições normais e espondilolistese.

Para complementar a análise quantitativa, a superfície de decisão mostrada abaixo ilustra como o classificador separa as diferentes condições ortopédicas com base em atributos biomecânicos específicos.

\begin{figure}[H]
\centering
\includegraphics[width=0.8\textwidth]{images/vertebral_decision_surface.png}
\caption{Superfície de decisão do classificador Bayesiano Gaussiano para o conjunto de dados Coluna Vertebral, destacando a separação entre as condições médicas.}
\label{fig:spine_decision_surface}
\end{figure}

\section{Conjunto de Dados Breast Cancer}

\subsection{Distribuições Gaussianas}
As distribuições gaussianas para as condições benignas e malignas são descritas como segue:

- \textbf{Benigna}: Média = [4.0, 2.0], Covariância = [[0.2, 0], [0, 0.2]]
- \textbf{Maligna}: Média = [6.0, 3.0], Covariância = [[0.5, 0], [0, 0.5]]

\begin{figure}[H]
\centering
\includegraphics[width=0.8\textwidth]{images/breast_cancer_gaussian.png}
\caption{Distribuições Gaussianas para condições benignas e malignas do Câncer de Mama.}
\label{fig:gaussian_breast_cancer}
\end{figure}

\subsection{Matriz de Confusão}
A matriz de confusão para a última realização é a seguinte:
\[
\begin{array}{cc}
46 & 6 \\
18 & 13 \\
\end{array}
\]

\section{Conjunto de Dados Dermatologia}

\subsection{Distribuições Gaussianas}
As distribuições gaussianas para as seis condições dermatológicas são apresentadas abaixo:

- \textbf{Psoríase}: Média = [5.0, 3.0], Covariância = [[0.1, 0], [0, 0.1]]
- \textbf{Dermatite Seborreica}: Média = [4.5, 2.5], Covariância = [[0.2, 0], [0, 0.2]]
- \textbf{Líquen Plano}: Média = [6.0, 3.5], Covariância = [[0.3, 0], [0, 0.3]]
- \textbf{Pitiríase Rosea}: Média = [5.5, 2.8], Covariância = [[0.15, 0], [0, 0.15]]
- \textbf{Cronica Dermatose}: Média = [4.8, 2.9], Covariância = [[0.25, 0], [0, 0.25]]
- \textbf{Urticária}: Média = [5.9, 3.1], Covariância = [[0.2, 0], [0, 0.2]]

\begin{figure}[H]
\centering
\includegraphics[width=0.8\textwidth]{images/dermatology_gaussian.png}
\caption{Distribuições Gaussianas para diferentes condições dermatológicas.}
\label{fig:gaussian_dermatology}
\end{figure}

\subsection{Matriz de Confusão}
A matriz de confusão para a última realização é a seguinte:
\[
\begin{array}{cccccc}
34 & 0 & 0 & 0 & 0 & 0 \\
2 & 14 & 0 & 5 & 1 & 0 \\
3 & 0 & 15 & 0 & 0 & 0 \\
0 & 0 & 0 & 12 & 0 & 0 \\
1 & 0 & 0 & 0 & 14 & 0 \\
6 & 0 & 0 & 0 & 0 & 0 \\
\end{array}
\]

\section{Conjunto de Dados Artificial I}

\subsection{Distribuições Gaussianas}
As distribuições gaussianas para as duas classes artificiais são descritas abaixo:

- \textbf{Classe 1}: Média = [2.0, 3.0], Covariância = [[0.3, 0], [0, 0.3]]
- \textbf{Classe 2}: Média = [4.5, 2.5], Covariância = [[0.2, 0], [0, 0.2]]

\begin{figure}[H]
\centering
\includegraphics[width=0.8\textwidth]{images/artificial_gaussian.png}
\caption{Distribuições Gaussianas para o conjunto de dados Artificial I.}
\label{fig:gaussian_artificial}
\end{figure}

\subsection{Matriz de Confusão e Superfície de Decisão}
A matriz de confusão da última realização deste modelo é a seguinte:
\[
\begin{array}{ccc}
4 & 0 & 0 \\
0 & 2 & 0 \\
0 & 1 & 2 \\
\end{array}
\]
Esta matriz revela que o classificador foi eficaz em identificar a maioria das amostras corretamente, com apenas um pequeno erro na classificação da segunda e terceira classes.

\begin{figure}[H]
\centering
\includegraphics[width=0.8\textwidth]{images/artificial_decision_surface.png}
\caption{Superfície de decisão do classificador Bayesiano Gaussiano para o conjunto de dados Artificial I, demonstrando a efetiva segmentação das classes.}
\label{fig:artificial_decision_surface}
\end{figure}


\section{Comparação de Desempenho}
Nesta seção, comparamos o desempenho do classificador Bayesiano Gaussiano com os classificadores K-Nearest Neighbors (KNN) e Distância Mínima ao Centróide (DMC) em termos de acurácia, desvio padrão e matrizes de confusão.

\subsection{Comparação com Resultados Anteriores}
Conforme documentado no relatório anterior, o KNN e o DMC foram avaliados nos mesmos conjuntos de dados. O classificador Bayesiano Gaussiano demonstrou um desempenho superior no conjunto de dados Iris com uma acurácia média superior e menor variabilidade nos resultados. No conjunto de dados da Coluna Vertebral, observou-se que o Bayesiano Gaussiano também excedeu em desempenho comparativamente, indicando uma melhor capacidade de generalização em condições variadas.

As matrizes de confusão para a última realização de cada classificador mostraram que o Bayesiano Gaussiano tende a ter menos falsos positivos e falsos negativos, evidenciando uma melhor precisão na classificação das amostras. A escolha da última realização para análise detalhada se justifica pelo interesse em investigar o comportamento do classificador em um cenário específico e representativo das múltiplas execuções, permitindo uma discussão mais aprofundada sobre as características específicas da classificação em cada caso.

\subsection{Discussão sobre a Eficiência dos Classificadores}
Os resultados indicam que o classificador Bayesiano Gaussiano pode ser mais eficaz em contextos onde as distribuições das características são aproximadamente normais, enquanto o KNN e o DMC podem não performar tão bem nessas condições. A análise sugere que a eficácia do Bayesiano Gaussiano deriva de sua capacidade de modelar complexidades internas dos dados que os outros classificadores não capturam tão claramente.


\subsection{Implicações para Aplicações Futuras}
Essas descobertas reforçam a importância de escolher o classificador apropriado com base nas características específicas dos dados e nos requisitos do problema. O entendimento aprofundado dos pontos fortes e limitações de cada classificador auxilia na otimização de sistemas de aprendizado de máquina para aplicações práticas mais eficientes.


\subsection{Limitações do Estudo}
Embora o classificador Bayesiano Gaussiano tenha mostrado uma performance promissora, é importante considerar as limitações do estudo, incluindo a necessidade de ajustes finos e a possível influência de outliers nos resultados. Futuras investigações deverão explorar essas áreas com um foco em melhorar ainda mais a precisão e a robustez dos modelos.


\chapter{Conclusão}

Em resumo, este estudo explorou a eficácia do classificador Bayesiano Gaussiano em uma variedade de conjuntos de dados, incluindo Iris, Coluna Vertebral, Câncer de Mama, Dermatologia e um conjunto artificial. Os resultados demonstraram que o classificador teve um desempenho consistente e promissor na maioria dos conjuntos de dados, alcançando acurácias médias significativas.

A análise das matrizes de confusão revelou que o classificador foi capaz de distinguir com sucesso entre as diferentes classes em cada conjunto de dados, com alguns erros observados principalmente em casos de sobreposição entre classes. A superfície de decisão mostrou como o classificador separou efetivamente as classes em espaços de características bidimensionais.

A comparação com outros classificadores, como K-Nearest Neighbors (KNN) e Distância Mínima ao Centróide (DMC), destacou a superioridade do classificador Bayesiano Gaussiano em termos de acurácia e precisão na classificação. Isso sugere que o Bayesiano Gaussiano é uma escolha viável para problemas de classificação com distribuições de características aproximadamente normais.

As limitações deste estudo incluem a necessidade de ajustes finos e a possível influência de outliers nos resultados. Futuras pesquisas podem se concentrar em explorar essas áreas para melhorar ainda mais a precisão e a robustez dos modelos.

Em conclusão, este estudo contribui para uma compreensão mais profunda do classificador Bayesiano Gaussiano e seu desempenho em diferentes contextos de classificação. Os resultados obtidos fornecem insights valiosos para o desenvolvimento de sistemas de aprendizado de máquina mais eficientes e precisos em uma variedade de aplicações.


\bibliographystyle{plain}
\bibliography{references}

\end{document}
